\chapter{Conclusions and Outlook}

The first conclusion, although it is actually almost a premise to this dissertation, is that sputtering is indeed an important effect that needs considering in high-fluence irradiation. This is especially true in the case of nanostructures where sputter yields can be greatly enhanced, as was shown this dissertation for the example of $Si$-nanowires. These results show that a good qualitative estimation, or intuition, of how any given nanostructure will be sputtered can be obtained using the Sigmund model for sputtering. The relative size of the overlap of the ions' nuclear energy loss and the surface of the target, even if it is nanostructured, is a reasonable estimation for the relative sputter yield. Thus, a feeling for which part of a complicated nanostructure will be most affected by sputtering during the irradiation with ions of a certain species and energy can be gained. For a specific nanoscale geometry the MC BCA simulation program \emph{iradina} \cite{borschel_ion_2011} can be used to make a more detailed analysis. The diameter dependence of sputter yield simulated with \emph{iradina} is qualitatively reproduced by the experiments performed in this thesis.

The quantitative values of sputtering are, in general, not accessible through the naive use of MC BCA simulations. However, very good agreement can be found for certain material and ion combinations \cite{biersack_computer_1987,hofsass_simulation_2014}, so the situation is not at all hopeless. The main difficulty is to find correct low energy interaction potential for the colliding atoms and ions for the given situation. As the secondary ion mass spectrometry (SIMS) technique is highly reliant on sputter yields and MD simulations also require the correct interaction potential at low particle energies, there is some interest in solving this problem. As sputtering is dominated by low energy collisions, it is very sensitive to the interaction potential precisely at the energy range where it is not easily accessible to other experiments. Therefore, experiments on sputtering of defined nanostructures, such as the ones performed on nanowires within this thesis, may be a useful approach to test theoretical predictions based on different interaction potentials. Such experiments should be combined and compared with ion impact-angle dependent measurements of the sputtering \cite{hofsass_simulation_2014} and the angle resolved emission of the sputtered atoms \cite{wirtz_storing_2008,verdeil_angular_2008}. This approach will not produce the correct interaction potential, however, it can be used to test results from simulations with different potentials to determine which describes the inter-atomic interaction best.

The main goal of ion irradiation is typically not sputtering, but the incorporation of dopants in the target. For nanostructured targets, care has to be taken to avoid an inhomogeneous irradiation and doping profile due to shadowing of the ion beam. This was illustrated with the nano-XRF investigation of $Mn$-doped $ZnO$ nanowires. This investigation shows that the BC MCA simulation is adequate for the prediction of the doping concentration for low ion fluences in nanowires. The limit of this applicability is given by the point where around $20\%$ of the material affected by the ion beam is sputtered, which in nanostructures is typically $20\%$ of the whole nanostructures' volume. Similar approximations can be made in bulk \cite{moller_tridyn_1984,andersen_computer_1986,moller_tridyn-binary_1988,sigmund_alloy_1993,zaporozchenko_preferential_1995}, even if the given references don't explicitly state a limiting fluence or sputtered depth. For irradiations with higher fluences, dynamic simulations are needed to predict the correct dopant concentration and profile. For nanostructures this ion fluence can be much lower than in bulk as there is less material to be sputtered and sputtering is enhanced. For high fluence irradiations, where more than $20\%$ of the material is expected to be sputtered, dynamic simulations are recommended. Software, which can dynamically change the structure and composition of the ion irradiated, nanostructured target, has been revealed recently in reference \cite{moller_tri3dyn_2014}. A comparison of the experimental results presented in this thesis with the results from such a simulation is a logical next step.

Down these lines, the application of the nano-XRF quantification technique to ion irradiated nanostructures can produce further interesting results. As nano-XRF is highly sensitive to elemental concentrations, it can widen the scope of the proposed studies into sputtering by investigating compositional changes ion irradiated nanostructures of compound materials. In compound materials preferential sputtering of one of the materials' components may become relevant even before a high dopant concentration has been reached. The interplay of nano-structuring, compositional changes and preferential sputtering could thus be investigated for a vast array of materials, by no means limited to semiconductors. Comparison to simulation results would further the understanding of the parameters influencing the preferential sputtering, which has practical meaning in secondary ion mass spectroscopy (SIMS), but also in the development of materials for fusion reactor components \cite{kelly_attempt_1978,roth_sputtering_1990,kenmotsu_effect_2011}. Using nanowires for such an experiment has the advantage that samples with multiple diameters can be fabricated in parallel and thus a larger parameter space becomes accessible to simultaneous investigation.

The sputtering of nanostructures is enhanced relative to bulk not only because of the high surface to volume ratio, but also by thermal effects. These can be very pronounced if the energy deposited by the impinging ion is confined to a small volume \cite{greaves_enhanced_2013,ilinov_sputtering_2014,nietiadi_sputtering_2014,anders_sputtering_2015,urbassek_sputter_2015}. This can lead to explosive ejection of large clusters of $1000s$ of atoms. Such extreme thermal sputtering effects are not observed in the experiments presented in this thesis; firstly, because the nanowires are relatively large compared to the ion energy deposited in them, leading, on average, to only little energy deposited per atom; secondly, the ion mass and target atom mass are both relatively low, leading to a low stopping power and a low density of the energy deposition. The simulation of sputtering with the BCA and the Sigmund theory would break down in experiments where this is not the case. Nevertheless, the maximum sputter yield observed at lower nanowire diameters in the experiment than in the simulation. This could be caused by increased sputtering due to larger thermal effects in thinner nanowires than in thicker ones.

Silicon nanowires show plastic deformation when irradiated with medium weight $Ar^+$ ions at room-temperature and energies of $100\,keV$. It could be shown that this deformation is not mediated by point defects and is not directed along the ion beam. Therefore, a surface tension driven model, which relies on a locally reduced viscosity, is presented. Where the lower threshold for the deformation is and whether there is an upper threshold energy above which the deformation ceases is not clear. It is clear, however, that the observed deformation is not in line with the ion track induced plastic deformation proposed by Trinkaus et al. \cite{trinkaus_viscoelastic_1995}, as the energy loss of the ion is too low and the locally increased pressure required by this model is not likely to be confined in the limited volume of a nanowire. The deformation is observed in amorphous $Si$-nanowires, but not in crystalline $Si$, both irradiated at elevated temperatures. In the crystalline case, the efficient recrystallization of the ion damaged nanowire volume recreates the long range order of the crystal lattice, while the amorphous material is free to remain deformed.

The plastic deformation of $Si$, highly localized at the point of the ions impact, has great potential for nanostructuring applications. It may be relevant to the formation of nanopores \cite{george_nanopore_2010} and certainly to the bending and manipulation of nanowires \cite{cui_ion-beam-induced_2013} and freestanding films \cite{kim_focused_2006}. It could be possible to go as far as building $Si$ nano-origami \cite{chalapat_self-organized_2013} with suitable templates. This may be a versatile tool in the growing field of $Si$ MEMS devices. Furthermore it may have to be considered in the formulation of a new mechanism for the formation of ripples on ion irradiated $Si$ surfaces. The dated model by Bradley and Harper \cite{bradley_theory_1988} considers only curvature and angle dependent sputtering as a roughening mechanism and is contested by models including ion induced strain and mass-transport \cite{norris_stress-induced_2012,kramczynski_wavelength-dependent_2014}. The latter shows some similarity with the results presented here, indicating that an atomistic investigation may be necessary to resolve this issue.

All three chapters of this thesis have compared MC BCA simulations performed with \emph{iradina} to experimental results on nanowires irradiated with high fluences. One limit to the applicability of the BCA is found were thermal effects have to be considered. This is not quite the case for the presented diameter-dependent sputter yields, where the simulation overlaps qualitatively with the experimental results. Only a slight shift in the diameter of maximum sputtering may indicate an influence of local, ion-induced heating. However, the plastic deformation found in amorphous $Si$-nanowires can not be explained with this simulation technique. Furthermore, the accuracy of the prediction of the doping concentration in nanostructures determined with \emph{iradina} simulations is satisfactory, but limited to low ion fluences. When the desired doping concentration is high and high ion fluences have to be implanted, dynamic simulations become neccessary. A rule of thumb lower limit to what constitutes a `high' fluence is given by the fluence at which $20\%$ of the volume effected by the ion beam is sputtered.
 


