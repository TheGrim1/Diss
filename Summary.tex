\chapter{Conclusions and Outlook}

The first conclusion, although it is actually almost a premise to this dissertation, is that sputtering is indeed an important effect that needs considering in high-fluence irradiation. As was shown in this dissertation for the example of $Si$-nanowires, this is especially true in the case of nanostructures, where sputter yields can be greatly enhanced. Qualitatively, a good estimation, or intuition, of how any given nanostructure will be sputtered can be obtained using the Sigmund model for sputtering. The relative size of overlap of the nuclear energy loss of the ion and the surface of the target, even if it is nanostructured, is a reasonable estimation for the relative sputter yield. Thus, a feeling for which part of a complicated nanostructure will be most affected by sputtering during the irradiation with ions of a certain species and energy can be gained. For a specific nanoscaled geometry the MC BCA simulation program \emph{iradina} \cite{borschel_ion_2011} can be used to make a more detailed analysis. The diameter dependence of the simulated sputter yield is confirmed by the experiments performed in this thesis.

The quantitative values of sputtering are, in general, not accessible through the naive use of MC BCA simulations. However, very good agreement can be found for certain material and ion combinations \cite{biersack_computer_1987,hofsass_simulation_2014}, so the situation is not at all hopeless. The main difficulty is to find correct low energy interaction potential for the colliding atoms and ions for the given situation. As the secondary ion mass spectrometry (SIMS) technique is highly reliant on sputter yields and MD simulations also require the correct interaction potential at low particle energies, there is some interest in solving this problem. \TODO{Finnland antwortet} As sputtering is dominated by low energy collisions, it is very sensitive to the interaction potential precisely at the energy range where it is not easily accessible to other experiments. Therefore experiments on sputtering of defined nanostructures, such as the ones performed on nanowires within this thesis, may be a useful approach to test theoretical predictions based on different interaction potentials. Such experiments should be combined and compared with angle dependent measurements of the sputtering \cite{hofsass_simulation_2014} and the angle resolved emission of the sputtered atoms \cite{wirtz_storing_2008,verdeil_angular_2008}. Unfortunately, this aproach will not produce the correct interaction potential, however, it can be used to test results from simulations with different potentials to determine which describes sputtering best.

The main goal of ion irradiation is typically not sputtering, but the incorporation of dopants in the target. For nanostructured targets, care has to be taken to avoid an inhomogeneous irradiation and doping profile due to shadowing of the ion beam. This was illustrated with the nano-XRF investigation of $Mn$-doped $ZnO$ nanowires. The first new result from this investigation is that the BC MCA simulation is adequate for the prediction of the doping concentration for low ion fluences in nanowires. The limit of this applicability is given by the point where around $20\%$ of the material affected by the ion beam is sputtered, which in nanostructures is typically $20\%$ of the whole nanostructures' volume. Similar approximations can be made in bulk \cite{moller_tridyn_1984,andersen_computer_1986,moller_tridyn-binary_1988,sigmund_alloy_1993,zaporozchenko_preferential_1995}, even if the given references don't explicitly state a limiting fluence or sputtered depth. For irradiations with higher fluences, dynamic simulations are needed to predict the correct dopant concentration and profile. For nanostructures this ion fluence can be much lower than in bulk as there is less material to be sputtered and sputtering is enhanced. Software, which can dynamically change the structure and composition of the ion irradiated, nanostructured target, has been revealed recently in reference \cite{moller_tri3dyn_2014}. A comparison of the experimental results presented in this thesis with the results from such a simulation is a logical next step.

Down these lines, the application of the nano-XRF quantification technique to ion irradiated nanostructures can produce further interesting results. As nano-XRF is highly sensitive to elemental concentrations, it can widen the scope of the proposed studies into sputtering by investigating compositional changes ion irradiated nanostructures of compound materials. In compound materials preferential sputtering of one of the materials' components may become relevant even before a high dopant concentration has been reached. The interplay of nano-structuring, compositional changes and preferential sputtering could thus be investigated for a vast array of materials, by no means limited to semiconductors. Comparison to simulation results would further the understanding of the parameters influencing the preferential sputtering, which has practical meaning in SIMS, but also in the development of materials for fusion reactor components \cite{kelly_attempt_1978,roth_sputtering_1990,kenmotsu_effect_2011}. Using nanowires for such an experiment has the advantage that samples with multiple diameters can be fabricated in parallel and thus a larger parameter space becomes accessible.

The sputtering of nanostructures is not only enhanced relatice to bulk because of the high surface to volume ratio, but may also be influenced by thermal effects, which are pronounced because energy deposited by the impinging ion is confined to a small volume \cite{greaves_enhanced_2013,ilinov_sputtering_2014,nietiadi_sputtering_2014,anders_sputtering_2015,urbassek_sputter_2015}. In the experiments presented in this thesis only minor thermal effects are observed. Two approximations show why this is the case: firstly, the nanostructures are relatively large compared to the ion energy deposited in them, leading to only a little energy deposited per atom; secondly, the ion mass and target atom mass are both relatively low, leading a low energy loss and a low density of the energy deposition.

Nevertheless, $Si$ nanowires show plastic deformation when irradiated with medium weight ($Ar^+$) ions at room-temperature and energies of $100$-$300\,keV$. Where the lower threshold for the deformation is and whether there is an upper threshold energy above which the deformation ceases is not clear. It is clear that the observed deformation is not in line with the ion track induced plastic deformation proposed by Trinkaus et al. \cite{trinkaus_viscoelastic_1995}, as the energy loss of the ion is to low. 

For high ion energies ($>\,1\,MeV$), the deformation mechanism may transition from the proposed 

Origami
deformation of layers
 


