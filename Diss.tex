\documentclass[11pt,
paper=a4,				
DIV=calc,		  % führt die Satzspiegelberechnung neu aus
%oneside,		  % einseitiger Druck
%BCOR=16mm,	  % Bindekorrektur
headinclude,
%footinclude
]{scrbook}


%****************
% define medatata
%________________
\def\Title{An Example Document}
\def\Author{Some Name}
\def\Subject{An Example Document}
\def\Keywords{LaTeX,Example,Document}
 
%***************************************************************************
% \convertDate converts D:20080419103507+02'00' to 2008-04-19T10:35:07+02:00
%___________________________________________________________________________
\def\convertDate{%
    \getYear
}
 
{\catcode`\D=12
 \gdef\getYear D:#1#2#3#4{\edef\xYear{#1#2#3#4}\getMonth}
}
\def\getMonth#1#2{\edef\xMonth{#1#2}\getDay}
\def\getDay#1#2{\edef\xDay{#1#2}\getHour}
\def\getHour#1#2{\edef\xHour{#1#2}\getMin}
\def\getMin#1#2{\edef\xMin{#1#2}\getSec}
\def\getSec#1#2{\edef\xSec{#1#2}\getTZh}
\def\getTZh +#1#2{\edef\xTZh{#1#2}\getTZm}
\def\getTZm '#1#2'{%
    \edef\xTZm{#1#2}%
    \edef\convDate{\xYear-\xMonth-\xDay T\xHour:\xMin:\xSec+\xTZh:\xTZm}%
}
 
\expandafter\convertDate\pdfcreationdate 
 
%**************************
% get pdftex version string
%__________________________
\newcount\countA
\countA=\pdftexversion
\advance \countA by -100
\def\pdftexVersionStr{pdfTeX-1.\the\countA.\pdftexrevision}
 
 
%*********
% XMP data
%_________
\usepackage{xmpincl}
%\includexmp{pdfa-1b}
 
%********
% pdfInfo
%________
\pdfinfo{%
    /Title    (\Title)
    /Author   (\Author)
    /Subject  (\Subject)
    /Keywords (\Keywords)
    /ModDate  (\pdfcreationdate)
    /Trapped  /False
}
 
 

\usepackage[utf8]{inputenc}
\usepackage{color}


\bibliographystyle{alpha} 
\newcommand{\fett}[1]{{\bf #1}}
\newcommand{\TODO}[1]{{\LARGE{\textcolor{red}{\emph {#1 }}}}}
\clubpenalty10000
\widowpenalty10000

\makeindex
\begin{document}

\titlehead{Friedrich Schiller Universität Jena\\
PAF}
\subject{Dissertation}
\title{High-Fluence Ion Irradiation of Nanostructured Semiconductors}
\author{Andreas Johannes}
\date{März 2015}
\maketitle


%\renewcommand{\abstractname}{Zusammenfassung}

\chapter*{Abstract}

Hier alles Bla


\tableofcontents

\chapter{Introduction}


Unser Zeitalter wird definierend durch Technologien der elektronischen Datenverarbeitung (EDV). Die EDV schlägt sich in vielen Bereichen nieder da sie umfangreiche Datenverarbeitung und Sammlung ermöglicht. Die Implantationen in dieser Arbeit hätte man noch auf einem analog gesteuerten Beschleuniger machen können, ein Rasterelektronenmikroskop (REM) ist auch noch analog vorstellbar, die Auswertung von hunderten REM Bildern ist jedoch schon an der Grenze des Möglichen. Des Weiteren ist die Möglichkeit ein Experiment wie dieses am Rechner zu simulieren grundlegend neu und aus der Wissenschaft nicht mehr wegzudenken. Des Weiteren hat die EDV schnelle Kommunikation über lange Strecken ermöglicht. Allein das Internet hat einen direkten und einschneidenden Einfluss auf alles, von den makroökonomischen Zusammenhängen, bis in das tägliche Lebens eines Einzelnen. Zwar ist der Welthandel v.A. durch den unwahrscheinlich ausgebauten Flug und Schiffsverkehr globalisiert, jedoch ist diese Globalisierung undenkbar ohne ein gleichzeitiges Entstehen eines globalen Informationsaustauschsystems. Über das Internet ist möglich sich \emph{zu Hause} über ein Reiseziel so gut wie irgendwo auf der Welt umfassend zu informieren, einen Flug in das entsprechende Land zu buchen, dort ein Hotel zu buchen, ein Taxi für den Ankunftszeitpunkt zu bestellen und im Restaurant das vegetarische Menü zu reservieren; und das alles gleich zu bezahlen. Vor zwei Generationen alles noch völlig undenkbar. Solche Beispiele lassen sich beliebig zusammenstellen und das Internet ist auch nur ein Teil der EDV, zu der auch andere prägende 

Die Größe der kleinsten strukturierten Bereiche in aktuellen integrierten Schaltkreisen wurde kontinuierlich reduziert und ist aktuell bei $22\,nm$.  , die 

\chapter{Methods}

Ausgiebige Verwendung von \emph{iradina} \cite{borschel_ion_2011}

\chapter{Sputtering of Nanostructures}

\chapter{High Doping Concentrations in Nanostructures}

\chapter{Plastic Flow in Nanostructured Silicon}

\chapter{Summary and Outlook}


\TODO{test}


\bibliography{DissBib}

\end{document}
