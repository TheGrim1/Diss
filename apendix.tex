
\newpage
\thispagestyle{empty}
\mbox{}

\begin{appendices}

%\titlespacing*{\chapter}{0pt}{-100pt}{40pt}

\chapter{List of Publications}

\setstretch{1.1}
\setlength{\parskip}{2em}
\setlength{\parindent}{0em}

Anomalous Plastic Deformation and Sputtering of Ion Irradiated Silicon Nanowires. \\
\textbf{Johannes A.}, Noack S., Wesch W., Glaser M., Lugstein A., Ronning C. \\
\emph{Nano Letters.} \textbf{15} (6):3800-07 (2015)


Ion beam irradiation of nanostructures: sputtering, dopant incorporation, and dynamic annealing.\\
\textbf{Johannes A.}, Holland-Moritz H., Ronning C.  \\
\emph{Semiconductor Science and Technology.} \textbf{30} (3):033001. (2015)


Enhanced sputtering and incorporation of Mn in implanted GaAs and ZnO nanowires.\\
\textbf{Johannes A.}, Noack S., Paschoal W., Kumar S., Jacobsson D., Pettersson H., Samuelson L., Dick K.A., Martinez-Criado G., Burghammer M., Ronning C.\\
\emph{Journal of Physics D-Applied Physics.} \textbf{47} (39):394003. (2014)


Persistent ion beam induced conductivity in zinc oxide nanowires. \\
\textbf{Johannes A.}, Niepelt R., Gnauck M., Ronning C.\\
\emph{Applied Physics Letters.} \textbf{99} (25):252105. (2011)

\newpage

Magnetic Polarons and Large Negative Magnetoresistance in GaAs Nanowires Implanted with Mn Ions. \\
Kumar S., Paschoal W., \textbf{Johannes A.}, Jacobsson D., Borschel C., Pertsova A., Wang C. H., Wu M. K., Canali C. M., Ronning C., Samuelson L., Pettersson H. \\ 
Nano Letters. \textbf{13} (11):5079–84. (2013)


Magnetoresistance in Mn ion-implanted GaAs:Zn nanowires.\\
Paschoal W., Kumar S., Jacobsson D., \textbf{Johannes A.}, Jain V., Canali C. M., Pertsova A., Ronning C., Dick K.A., Samuelson L., Pettersson H. \\
\emph{Applied Physics Letters.} \textbf{104} (15):153112. (2014)

Atomic-scale structure, cation distribution, and bandgap bowing in Cu(In,Ga)S$_2$ and Cu(In,Ga)Se$_2$. \\
Eckner S., Kämmer H., Steinbach T., Gnauck M., \textbf{Johannes A.}, Stephan C., Schorr, S., Schnohr, C.S. \\
\emph{Applied Physics Letters.} \textbf{103} (8):081905. (2013) 

Improved Ga grading of sequentially produced Cu(In,Ga)Se$_2$ solar cells studied by high resolution X-ray fluorescence.\\
Schöppe P., Schnohr C. S., Oertel M., Kusch A., \textbf{Johannes A.}, Eckner S., Burghammer M., Martinez-Criado G., Reislöhner U., Ronning C.\\
\emph{Applied Physics Letters.} \textbf{106} (1):013909. (2015) 

\chapter{List of oral and poster presentations}

\setlength{\parskip}{1em}

Poster: ``Persistent ion beam induced conduction"\\
DPG Frühjahrstagung, Berlin 2012

Poster: ``Wiring quantum dots"\\
$6^{th}$ Nanowire Growth Workshop, St. Petersburg 2012

Poster: ``Ion beam tailoring of nanowire diameters"\\
International Conference on One-imensional Nanomaterials (ICON), Annecy 2013

Poster: ``Ion beam tailoring of nanowire diameters"
Focus Workshop Nanowires, Munich 2013

Talk: ``X-Ray nano probe of implanted semiconductor nanowires"\\
European Synchrotron Radiation Facility, User Meeting, Grenoble 2014

Talk: ``Enhanced sputtering and incorporation of Mn in implanted GaAs and ZnO nanowires"\\
Ionenstrahlen und Nanustrukturen Workshop, Paderborn, 2014

Poster: ``Enhanced sputtering and incorporation of Mn in implanted GaAs and ZnO nanowires"\\
Ion Beam Modification of Materials, Leuven 2014

Talk: ``High Fluence Ion Implantation in Nanowires''\\
Helmholtz-Zentrum Dresden Rossendorf 2015


\setlength{\parskip}{0em}
\setlength{\parindent}{0.3cm}

\newpage
\thispagestyle{empty}
\mbox{}


\chapter{Ehrenwörtliche Erklärung}

Ich erkläre hiermit ehrenwörtlich, dass ich die vorliegende Arbeit selbständig, ohne unzulässige Hilfe Dritter und ohne Benutzung anderer als der angegebenen Hilfsmittel und Literatur angefertigt habe. Die aus anderen Quellen direkt oder indirekt übernommenen Daten und Konzept sind unter Angabe der Quelle gekennzeichnet.

Bei der Auswahl und Auswertung des gezeigten Materials haben mir die nachstehend aufgeführten Personen in der jeweils beschriebenen Weise entgeltlich/unentgeltlich geholfen:

\begin{enumerate}
\item{Die Aufnahme und die halb-automatisierte Auswertung der SEM Bilder zum Zerstäuben und zum plastischen Fließen in Si Nanodrähten wurden von Stefan Noack im Rahmen seiner Masterarbeit gemacht.}
\item{Die geätzten und VLS-gewachsenen Si Nanodrähte hat Markus Glaser in Wien hergestellt. Von ihm habe ich auch die SEM Bilder der VLS Drähte vor und nach der Bestrahlung mit In und As.}
\item{Das plastische Fließen in Si Nanodrähten wurde insbesondere mit Alois Lugstein, der die Idee der Bestrahlung `andersherum‘ hatte, und Prof. Werner Wesch diskutiert.}
\item{Die Vorgehensweise bei der PyMCA Auswertung wurde von Gema Martinez-Criado übernommen.}
\item{Die Berechnung der Redeposition in Kapitel \ref{sec:redeposition} wurde zusammen mit Emanuel Schmidt gemacht.}
\item{Die jeweiligen Koautoren der angeführten Publikationen waren an der Erstellung und Interpretation der Ergebnisse und deren Darstellung beteiligt.}
\end{enumerate}

Weitere Personen waren an der inhaltlich-materiellen Erstellung der Vorliegenden Arbeit nicht beteiligt. Insbesondere habe ich hierfür nicht die entgeltliche Hilfe von Vermittlung- bzw. Beratungsdienten (Promotionsberater oder andere Personen) in Anspruch genommen. Niemand hat von mir unmittelbar oder mittelbar geldwerte Leistungen für Arbeiten enthalten, die im Zusammenhang mit dem Inhalt der vorgelegten Dissertation stehen.

Die Arbeit wurde bisher weder im In- noch im Ausland in gleicher oder ähnlicher Form einer anderen Prüfungsbehörde vorgelegt.

Die geltende Prüfungsordnung der Physikalisch-Astronomischen Fakultät ist mir bekannt.
Ich versichere ehrenwörtlich, dass ich nach bestem Wissen die reine Wahrheit gesagt und nichts verschwiegen habe.

\vspace{1.5cm}
Jena, den 09. Juli 2015 

\begin{flushright}
 Andreas Johannes
\end{flushright}

\setstretch{1.3}
\chapter{Lebenslauf}

\begin{tabular}{ll}
& Johannes, Andreas Walter\\
14.03.1986 & Geburtsort: Pretoria (RSA)\\
1992-2005 & Deutsche Schule Pretoria\\
2005 & Abitur\\
2006-2011 & Physikstudium an der FSU Jena\\
2011 & Hochschulabschluss: Physik Diplom\\
2011-2015 & Wissenschaftlicher Mitarbeiter an der FSU Jena, AG Ronning

\end{tabular}

\vspace{1.5cm}
Jena, den 09. Juli 2015 

\begin{flushright}
 Andreas Johannes
\end{flushright}


\setstretch{1.1}
\chapter{Danksagung}

Ich möchte mich bei Allen bedanken die in irgendeiner Weise mich während meines Aufenthalts im ``Roten Haus'' unterstützt haben. Angefangen natürlich bei Carsten Ronning: Du hast meinen Wissenschaftliche Werdegang tatsächlich schon sehr früh beeinflusst, obwohl ich nie eine Vorlesung bei dir gehört habe. Angefangen hab ich nämlich meine Studienarbeit bei Thomas Hahn und Heiner Metzner, aber weil ich etwas länger gebraucht habe, habe ich sie dann beendet unter Anleitung von Jacob Haarstrich und Carsten Ronning. Ich durfte als `kleiner Student' schon recht viele Geräte selber bedienen, eigene Ideen einbringen, diskutieren und ausprobieren. Das hat so viel Spaß gemacht, dass es mir nicht schwer viel die nächsten 5 Jahre weiter in deiner Arbeitsgruppe zu bleiben. Mir hat die produktive Atmosphäre, die du mit der offenen wissenschaftlichen Diskussion in den regelmäßigen Gruppentreffen und zwischendurch pflegst, sehr gefallen. Auch für die Möglichkeit, dein Ermuntern und die Unterstützung auf Konferenzen und Projektreffen die eigene Arbeit vorzustellen und sich mit diversen Wissenschaftlern auszutauschen bin ich sehr dankbar. Am wichtigsten war es mir, dass ich meinen Ideen und Interessen folgen konnte. Diese Arbeit ist zwar im Project ``wiring quantum dots'' entstanden, aber als sich herausstellte, dass ZnO für diesen Ansatz vllt. nicht so geeignet ist, hast du nicht darauf bestanden, dass ich das Project weiter verfolge, sondern hast mich `machen lassen‘ und mit den Kontakten zum ESRF dafür auch noch eine wunderbare experimentelle Methode an Land gezogen!

At this point many thanks to Gema Martinez-Criado and Manfred Burghammer with your respective teams at the ESRF. I was privileged to work with you and the excellent equipment you have set up and taught us to use fairly independently. Thanks for the opportunity, and your help and trust.

Thanks to the partners in the “wiring quantum dots” project. Anna, Yannik and Sonja. Thank you for the discussions, the nicest TEM investigations I have seen, which unfortunately didn’t make it into this thesis, and your friendly hospitality whenever I was in Lausanne. 

Auch an Alois und Markus einen Herzlichen Dank für die intensive Zusammenarbeit. Es war sehr hilfreich mit euch zu diskutieren und hat auch sehr viel Spaß gemacht euch mehrmals in Wien zu besuchen. Ohne deine Proben, Markus, und die Idee mit dem Bestrahlen von der Rückseite der Nanodrähte, Alois, hätte diese Arbeit ganz anders aussehen müssen.

Ich will denen danken deren Abschlussarbeiten ich betreuen dürfte, die mir also als fähige Experimentatoren (sprich: manchmal Messsklave) zur Seite standen: Benjamin, Kevin und Sven möchte ich danken, dass ihr es euch angetan habt bei mir die B.Sc. Arbeit anzufertigen, es lag an den diffizilen Materialeigenschaften, nicht an euern Bemühungen, dass ich eure Untersuchungen nicht weiter verfolgt habe. Matthias, du hast mit äußerstem Fleiß, sogar illegal viele Runs durch den 3ZJ gejagt und damit nicht nur mich mit guten Proben versorgt. Stefan, im Prinzip ist deine ganze Masterarbeit in dieser Dissertation wieder zu finden. Danke für die gute Arbeit!

Den Mitarbeitern der verschiedenen Werkstätten, insbesondere der Feinmechanik im gelben Haus, möchte ich für die unkomplizierte und oft sehr spontane Zusammenarbeit danken. Weiterhin den Kollegen aus der Konstruktion für die sehr schön gewordene Umsetzung der RHT-Stage.

Vielen Dank an Alle, die in irgendeiner Weise an der Herstellung von Proben mit denen ich weiterarbeiten konnte beteiligt waren. Helena Franke für die PLD-gewachsenen ZnO Proben, Markus Glaser für die VLS und geätzten Si Nanodrähte, Michael Oertel für das Aufsputtern von AZO und i-ZnO und Allen, die sich am 3ZJ und HTJ Mühe gemacht haben, dort insbesondere Matthias Ogrisek.

Ganz besonders möchte ich Carmen, Frank, Ulli und Patrick danken, dass ihr es immer wieder mit neunen, dummen Studenten aufnimmt ihnen alles zeigt, helft und zur Seite steht und auch dann noch geduldig seid, wenn sie irgendwann als möchte-gern Wissenschaftler alles besser wissen wollen. Besonders möchte ich mich auch bei Gerald bedanken, dass du mir die Bedienung des ROMEO gezeigt hast und mich ein Stück weit in die Geheimnisse seiner Quellen eingewiesen hast.

Aus Angst jemanden zu vergessen, kann ich hier nicht alle aktiven und ehemaligen Insassen des ``Roten Hauses‘‘ erwähnen, aber mit Raphael, Sebastian, Christian (zusätzlich, insbesondere und explizit auch wegen \emph{iradina}!), Jana, Martin, Christoph, Stefanie, Davide, Claudia, Phillip, Steffen, Jura, Anja, Marie hat es Spaß gemacht Zeit zu verbringen. Wenn nichts Anderes, so hat wenigstens die Kaffeerunde morgens mich motiviert rechtzeitig ins Institut zu kommen. Speziell aber möchte ich den langweiligen Kollegen aus dem Büro 203 dafür danken, dass ihr mich nie mit irgendeinem Quatsch abgelenkt habt, meine Ideen lieber ignoriert habt, als sie mit eigenen Vorschlägen durcheinander zu bringen und auch niemals für ein Spaß oder Bier zu haben ward. Dafür, dass ihr das Büromotto ``Arbeitszeit ist Leistungszeit‘‘ lebt, danke.

Meinen Eltern möchte ich für die stetige Unterstützung danken. Ihr habt mir eine außergewöhnliche Schulbildung und ein Studium nach Belieben ermöglicht, danke! Papa, ohne deine geduldigen Erklärungen zu allen möglichen Fragen in den naturwissenschaftlichen Fächern und Mathe wäre mein Interesse irgendwann mal `Forscher‘ zu werden vllt. nie geweckt.
 

\end{appendices}
