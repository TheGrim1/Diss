


%We are living in an era dominated by the information technology. There is virtually no part of life not influenced by the continuing advances in the digital world and semiconductors, especially silicon, are at the base of each and every logic unit dealing in `ones' and `zeros'. Silicon therefore quickly became one of the most studied materials, catching up with the previously dominating iron and its various alloys with other elements, steel.

Technological progress generally shows a competition between the optimization of the dominating technology and the development of fundamentally new operation principles. An example of this competition is recorded in the ``International Technology Roadmap for Semiconductors'', which aims to guide the scaling of digital devices to follow ``Moore's Law'' of improved performance, and the white paper ``Towards a `More-than-Moor' roadmap'', which examines opportunities to include non-digital functionality where performance does not necessarily have to scale with size. Both are available at the ITRS website  \cite{map_http://www.itrs.net/_2015}. A shift in operating principle is found, for example, in data storage, which changed fundamentally when the effect of giant magneto-resistance (GMR) was discovered in 1988 \cite{baibich_giant_1988,binasch_enhanced_1989}. This quickly formed the basis for the standard hard-drives (HDD) and HDDs soon dominated PC data storage. Nowadays, the older principle of flash memory is making a comeback in solid state drives (SSD), which begin to replace HDDs. They owe their viability (cost, speed and storage density) almost entirely to the advanced miniaturization, allowing the production of a floating gate for a transistor on a scale down to tens of nanometers per single $bit$, while producing \emph{billions} of $\nicefrac{bits}{cm^{2}}$ \cite{micheloni_inside_2013}. This shows, that it is not \emph{a priori} possible to discern with certainty which approach is going to produce the best results, so that much room is left for open minded fundamental research in general and on semiconductors in particular.

In the wake of the miniaturization aimed at the improvement of IT hardware technology, the new, multi-disciplinary field of nanotechnology has emerged. The scope of the field is illustrated by the high number of journals dedicated to research at the nanoscale. This includes semiconductor science, but in the leading journals \emph{ACS Nano} \cite{acs_nano_http://pubs.acs.org/journal/ancac3_2015}, \emph{Advanced Materials} \cite{advanced_materials_http://www.advmat./_2015}, \emph{Advanced Functional Materials} \cite{advanced_functional_materials_http://onlinelibrary.wiley.com/journal/10.1002/issn1616-3028_2015}, \emph{Nano Letters} \cite{nano_letters_http://pubs.acs.org/journal/nalefd_2015}, \emph{Nature Nanotechnology} \cite{nature_nanotechnology_http://www.nature.com/nnano_2015}, \emph{Nano Today} \cite{nano_today_http://journals.elsevier.com/17480132/nano-today/_2015}, \emph{Nanotoxicology} \cite{nanotoxicology_http://www.informahealthcare.com/nan_2015}, \emph{Small} \cite{small_http://www.small-journal.com/_2015} and others, fundamental research and applications of nanoscale devices and effects from all natural sciences are published.

The specific class of nanomaterials investigated within this thesis are semiconductor nanowires, which have gained a significant amount of  interest \cite{huang_room-temperature_2001,cui_nanowire_2001,duan_indium_2001,xia_one-dimensional_2003,lieber_functional_2007}. `Nanowire' is a term used for many morphologies, but it seems a reasonable name for structures with a cross-section that is between $1 \times 1$ and $1000 \times 1000\,nm^2$ and a large length to form a high aspect ratio. One of the general aspects of this shape and also of nanostructured materials in general, is that the surface properties play a dominating role. This is simply caused by the fact, that the surface-to-volume ratio is large. Because this ratio in general is proportional to $1/r$ for a body with a characteristic constraining length of $r$, it becomes very large for small structure sizes. The wire shape has an inherent advantage over three dimensionally constrained particles (nanoclusters, quantum dots etc.), in that it is easier to define contacts and drive a current through a nanoscale wire than through a nanoscale dot. 

%Incidentally, the idea to combine this specific advantage of nanowires with new properties obtained by the stronger three dimensional confinement of quantum dots is the main idea behind the `Deutsche Forschungsgemeinschaft' (DFG) project ``wiring quantum dots", which funded this work. 

The high surface to volume ration ensures that a semiconductor nanowire with two contacts at either end is already a very sensitive device. Such simple devices have been shown to be possible gas-sensors \cite{shen_devices_2009}, they could measure the $pH$ inside a cell \cite{cui_nanowire_2001} or the impact of a single ion \cite{johannes_persistent_2011}. By functionalization of the surface, it is, as one example of many, possible to create a selective biological sensor that can even detect the attachment of a single viruses \cite{patolsky_electrical_2004}. Additional functionality can be gained by adding impurities to semiconductor nanowires, because this dramatically changes their electronic properties \cite{sze_physics_2006}. For example, changing the doping from $n$ to $p$-type within a nanowire creates a $pn$-junction that can be used as a solar cell \cite{kempa_single_2008,christesen_design_2012}, while the combination of $n$ and $p$-type nanowires can be used to fabricate a thermo-electric generator \cite{schierning_silicon_2014}.

A particularity flexible method to add impurities to semiconductors is ion beam irradiation, because it can be used to `mix' (i.e. dope) virtually any target material with a precisely controlled number of atoms of practically any element. Ion beam doping is a well established technology, and it was and is a key part in the processing and development of semiconductor technologies \cite{hamm_industrial_2012}. In general, ion beam doping has the advantage over doping during the synthesis of nanostructures, because it is not inherently limited by the chemical potentials and thermodynamics, which typically have to be carefully controlled for the synthesis of nanostructures. It is a non-equilibrium physical process by which different elements can forcefully be introduced into a target matrix with much higher energies than those involved in chemical bonding. The extent of disorder created in the target during this bombardment, whether the intermixing is thermodynamically stable and whether a desired (crystalline) order can be reestablished by thermal annealing is in the focus of ion-beam physics. A good background on this can be gained from dedicated literature \cite{ziegler_stopping_1985,eckstein_computer_1991,nastasi/mayer/hirvonen_ion-solid_2008,schmidt_ion_2012}.

%Scaling down material to the nanoscale becomes interesting when the structure size reaches the characteristic dimension of the effect governing the physical properties of interest. Then new, mesoscopic and quantum mechanical properties can emerge. As an example, the changed light-matter interaction in structures of the dimension of the wavelength of light is investigated in the field of nano-photonics and plasmonics \cite{saleh_fundamentals_2007,maier_plasmonics:_2007}

Typical ion ranges for the doping of semiconductors lie in the range of $10$-$100\,nm$. Therefore, ion beam irradiation of nanostructures of the same dimension will show some interplay between the irradiated structures' dimensions and the ion range. The many practical applications of the combination of ion beams and nanostructures warrants general investigations of the nanostructure - ion beam interaction and the topic has therefore gained increased interest very recently \cite{borschel_ion-solid_2012,greaves_enhanced_2013,nietiadi_sputtering_2014,johannes_ion_2015,urbassek_sputter_2015}.  

A specific example in which the combination of nanostructures and ion beams is advantageous is the ion irradiation of diamond to create nitrogen-vacancy clusters \cite{babinec_diamond_2010}. The diamond is nanostructured to facilitate efficient extraction of light, while ion irradiation with nitrogen creates nitrogen-vacancy clusters very effectively. These are promising components in a future quantum information device. The precisely controlled ion irradiation, makes it possible to implant a well defined number of ions with reasonable spacial accuracy. This control is extravagantly demonstrated by the possibility of single ion irradiation \cite{meijer_concept_2006,ohdomari_single-ion_2008}. 

In addition to this extremely low ion fluence example of ion irradiation, the next two examples of the concurrence of nanotechnology and ion-irradiation led more or less directly into the fundamental investigations of high fluence irradiation presented in this dissertation. Firstly, there is the search for a nanostructured diluted magnetic semiconductor for which $Mn$ doped $GaAs$-nanowires are a good candidate. Because $GaAs$ nanowires typically grow above $450^\circ C$ but $MnAs$ segregates from $Ga_{(1-x)}Mn_xAs$ at around $350^\circ C$ \cite{dietl_engineering_2006,sadowski_gaasmnas_2011}, there is no straightforward way to dope $GaAs$-nanowires with high concentrations of $Mn$ during their growth. However, it can be achieved by implanting $Mn$ in $GaAs$-nanowires. Best results are achieved when the irradiation is performed at elevated temperatures, hot enough to minimize disorder introduced by the ion beam, but cold enough to prevent segregation of $MnAs$ \cite{borschel_new_2011,paschoal_hopping_2012,borschel_ion-solid_2012,kumar_magnetic_2013,paschoal_magnetoresistance_2014}. 

Conversely, the ``wiring quantum dots'' project, through which this thesis was funded, aimed to utilize the segregation of ion-implanted material in a nanowire to form nanowires decorated with nanoclusters. When $Si$ nanowires are irradiated with high fluences of $In^+$ and $As^+$ and subsequently annealed with a flash-lamp, separated $InAs$ slices form within the $Si$ nanowires \cite{prucnal_iii-v_2014,glaser_personal_2015}. The supersaturation of $Si$ with $In$ and $As$ by ion implantation can thus be utilized to create $Si$-$InAs$ nanowire hetero-structures from a $Si$ nanowire template in a relatively straightforward manner.
 
A further important example of the intersection of nanotechnology and ion beams is found in the ubiquitous focused ion beam (FIB) systems. The production and development of many of the novel nanoscale devices on the horizon often requires the precise ion beam milling that FIBs provide with a resolution of few nanometers \cite{kranz_integrating_2001,george_nanopore_2010,chalapat_self-organized_2013}. In all the examples given so far, and virtually per definition in the last one, the typical structure sizes irradiated are in the same order of magnitude at the range of the impinging ions. Understanding how this affects the ion-matter interaction can be crucial to the successful outcome of the respective experiments.

In the effort to understand principles and fundamental interactions on the nanometer length scales, nanowires are a very good model system to investigate, because their geometry is fully characterized by their height and radius. Spheres, which would have a degree of freedom less, are more difficult to handle, because the unavoidable proximity of a substrate may influence their behavior \cite{hu_burrowing_2002,klimmer_size-dependent_2009,moller_tri3dyn_2014,johannes_ion_2015}. The understanding of the ion-nanostructure interaction gained by investigating irradiated nanowires is principally transferable to any nanostructure. However, this can hardly be done in a general way explicitly, because the possible shapes of nanostructures are uncountable.

This dissertation will begin with an overview of the ion-solid interaction focused on the accuracy of simulations of this interaction. A detailed review of effects and literature relevant to the ion irradiation of nanostructured materials is also given (Chapter \ref{sec:ionsolid}). Next, the methods used for the experiments are briefly outlined (Chapter \ref{sec:experimental}). This dissertation adds to the growing field of nanostructure - ion beam interaction by discussing three effects which are especially important in high ion fluence irradiation. A separate chapter is dedicated to each high ion fluence experiment. 
 
\begin{description}
  \item[\normalfont Chapter \ref{sec:sputter} - Sputtering of Nanowires] \hfill \\
  In the dissertation of Dr. C. Borschel \cite{borschel_ion-solid_2012} the program \emph{iradina} \cite{borschel_ion_2011} was developed and used to simulate the ion irradiation of nanostructures. It predicts an enhanced, diameter-dependent sputter yield in nanostructures. Chapter \ref{sec:sputter} discusses the simulation and compares its predictions with experimentally obtained diameter-dependent sputtering in nanowires. Some first results on the sputtering during $Mn$ irradiation of $GaAs$-nanowires are published elsewhere \cite{johannes_enhanced_2014}. The results presented here are on $Ar$ irradiated $Si$-nanowires. They were obtained in close cooperation with Stefan Noack \cite{noack_sputter_2014} in his M.Sc. and also published in reference \cite{johannes_anomalous_2015}.
  \item[\normalfont Chapter \ref{sec:high} - High Doping Concentrations in Nanowires] \hfill \\
  The concentration of dopants does not follow a linear increase with the fluence of ions implanted for high fluences. It has already been observed that sputtering of the target will dynamically change its composition during the ion irradiation in addition to the intended change by incorporation of the ions within the target material \cite{moller_tridyn_1984,moller_tridyn-binary_1988,miyagawa_computer_1991,sigmund_alloy_1993,eckstein_oscillations_2000}. This effect is enhanced in nanostructures, first, because the sputtering is enhanced when compared to bulk samples, as demonstrated in Chapter \ref{sec:sputter}, but also because there is simply less material. Hence, the effect of removing material by sputtering already becomes significant at lower fluencies in nanostructures than in bulk. The presented results were acquired by compositional analysis using nano-XRF performed on 175\,keV Mn$^+$ ion irradiated ZnO nanowires and are partially published in reference \cite{johannes_enhanced_2014}. They are discussed in comparison to a pseudo-dynamic simulation performed using results from \emph{iradina}.
  \item[\normalfont Chapter \ref{sec:plastic} - Plastic Flow in Silicon Nanowires] \hfill \\
  In high ion fluence irradiated $Si$ nanowires an unexpected tendency of the nanowires to become shorter was observed. Chapter \ref{sec:plastic} presents a dedicated investigation into this plastic deformation of $Si$ under ion irradiation which has been previously seen only in high energy ($\ge\,MeV$) ion irradiations \cite{volkert_stress_1991,trinkaus_viscoelastic_1995,hedler_amorphous_2004,hedler_boundary_2005}. These results were also obtained in part within the M.Sc of Stefan Noack \cite{noack_sputter_2014} and are published in reference \cite{johannes_anomalous_2015}. A probable mechanism for the deformation can be presented by comparing the experimentally observed deformation with dedicated \emph{iradina} simulations and literature on MD simulations of similar conditions.
\end{description}
