


%We are living in an era dominated by the information technology. There is virtually no part of life not influenced by the continuing advances in the digital world and semiconductors, especially silicon, are at the base of each and every logic unit dealing in `ones' and `zeros'. Silicon therefore quickly became one of the most studied materials, catching up with the previously dominating iron and its various alloys with other elements, steel.

In technological advances, there is generally a competition between the optimization of the dominating technology and the development of fundamentally new operation principles. This competition can be found in the ``International Technology Roadmap for Semiconductors'', which aims to guide the scaling of devices to follow ``Moore's Law'' of improved performance, and the white paper ``Towards a ``More-than-Moor'' roadmap'' which examines opportunities to include non-digital functionality where performance don't necessarily have to scale with size (both available at the ITRS website  \cite{itrs_road_map_itrs_2015}). An example for a shift in operating principle for data storage was the fundamentally new effect of giant magneto-resistance (GMR) discovered in 1988. This quickly formed the basis for the standard hard-drives (HDD) and quickly dominated PC data storage. Nowadays, the much older principle of flash memory is making a come-back in solid state drives (SSD), which are beginning to replace HDDs. They owe their viability (cost, speed and storage density) almost entirely to the advanced miniaturization, allowing the production of a floating gate for a transistor on a scale down to tens of nanometers per single $bit$, while producing \emph{billions} of $bits/cm^{2}$. \emph{A priori} it is not possible to discern with certainty which approach is going to produce the best results, so that much room is left for open minded fundamental research.

As a side effect of miniaturization being a major factor in the improvement of all IT-hardware-technology, nanotechnology became somewhat of a buzzword. Fueled by this upwind for everything `nano', a peculiar class of materials gained a lot of academic interest: nanowires \cite{huang_room-temperature_2001,cui_nanowire_2001,duan_indium_2001,xia_one-dimensional_2003,lieber_functional_2007}. `Nanowire' is a term used for many morphologies, but it seems a reasonable name for structures with a cross-section that is between $1 \times 1$ and $1000 \times 1000\,nm^2$, which are significantly longer than they are wide. One of the general aspects of this shape and also of nanostructured materials in general, is that the surface properties play a dominating role. This is simply caused by the fact, that there is a lot of surface per volume of material. As the surface to volume ratio in general is proportional to $1/r$ for a body with a characteristic constraining length of $r$, it gets very large for small structure sizes. Investigating nanowires as catalysts or sensing devices tries to take advantage of this large active surface area. The wire shape in particular has an inherent advantage here over three dimensionally constrained particles (nanoclusters, quantum dots etc.), in that it is easier to define contacts and drive a current through a nanoscaled wire than through a nanoscaled dot. The idea to combine this specific advantage of nanowires with new properties obtained by the stronger three dimensional confinement of quantum dots is the main idea behind the `Deutsche Forschungsgemeinschaft' (DFG) project ``wiring quantum dots" which funded this work. 

Having somewhat motivated the use of semiconductor nanowires and before going into further detail on this specific project, first the ``ion beam irradiation'' part of the tile also needs an introduction. Although $Si$ is the material of choice for microchips (hence ``Silicon Valley''), as it is, pure silicon is a rather uninteresting material. The defining property of semiconductors is the ability to dramatically change their electronic properties by adding impurities \cite{sze_physics_2006}. As ion beam irradiation can be used to `mix' (i.e. dope) virtually any target material with a precisely controlled number of atoms of practically any element, it was and is a key part in the processing and development of semiconductor technologies. 

In general, ion beam doping has the advantage over doping during the synthesis of nanostructures, in that it is not inherently limited by the chemical potentials and dynamics which typically have to be carefully controlled for the synthesis of nanostructures. It is a non-equilibrium physical process by which different elements can forcefully be introduced into a target matrix with much higher energies than those involved in chemical bonding. The extent of disorder created in the target during this bombardment, whether the intermixing is thermodynamically stable, and whether a desired (crystal) order can be reestablished by thermal annealing is in the focus of ion-beam physics. A good background on this can be gained from dedicated literature \cite{ziegler_stopping_1985,eckstein_computer_1991,nastasi/mayer/hirvonen_ion-solid_2008,schmidt_ion_2012}.

A specific example in which the combination of nanostructures and ion beams is advantageous is the ion irradiation of diamond to create nitrogen-vacancy clusters. These are interesting as promising components in a future quantum information device \cite{babinec_diamond_2010}. The precise control ion irradiation gives, makes it possible to implant a well defined number of ions with reasonable spacial accuracy. This control is extravagantly demonstrated by the possibility of single ion irradiation \cite{meijer_concept_2006,ohdomari_single-ion_2008}. 

In addition to this extremely low ion fluence example of ion irradiation, the next two examples of the concurrence of nanotechnology and ion-irradiation led more or less directly into the investigations into high fluence irradiation presented in this dissertation. First is the search for a diluted magnetic semiconductor by implanting $Mn$ in $GaAs$ nanowires. As $GaAs$ nanowires typically grow above $450^\circ C$ but $MnAs$ segregates from $Ga_{(1-x)}Mn_xAs$ at $350^\circ C$ \cite{dietl_engineering_2006,sadowski_gaasmnas_2011}, there is no straightforward way to dope $GaAs$ with high concentrations of $Mn$ during nanowire growth. The key to this problem is to do the irradiation at elevated temperatures, hot enough to minimize disorder, but cold enough to prevent segregation of $MnAs$ \cite{borschel_new_2011,paschoal_hopping_2012,borschel_ion-solid_2012,kumar_magnetic_2013,paschoal_magnetoresistance_2014}. Conversely, in the before mentioned ``wiring quantum dots'' project the segregation of the implanted material was actually utilized to combine nanowires with nanoclusters. When $Si$ nanowires are irradiated with high fluences of $Ga$ and $As$ and subsequently annealed with a flash-lamp, separated $GaAs$ slices form within the $Si$ nanowires \cite{prucnal_iii-v_2014,glaser_personal_2015}. The supersaturation of $Si$ with $Ga$ and $As$ by ion implantation can thus be utilized to create $GaAs - Si$ nanowire hetero-structures from a $Si$ nanowire template in a relatively straightforward manner.
 
A final example of the intersection of nanotechnology and ion beams is found in the ubiquitous focused-ion-beam (FIB) systems. The production and development of many of the novel applications of nanostructures on the horizon often requires the precise ion-beam milling that FIBs provide with a resolution of few nanometers. In all the examples given so far, and virtually per definition in the last one, typical structure sizes irradiated are in the order of magnitude of the irradiating ions. In the effort to understand principles and fundamental interactions on the nanometer length scales, nanowires are a very good model system to investigate, as their geometry is fully characterized by their height and radius. Spheres, which would have a degree of freedom less, are unfortunately more difficult to handle, as the unavoidable proximity of a substrate may influence their behavior \cite{moller_tri3dyn_2014,johannes_ion_2015}. The understanding of the ion-nanostructure interaction gained by investigating irradiated nanowires is principally transferable to any nanostructure. However, this cannot of course be done in any general way explicitly, as the possible shapes of nanostructures are uncountable.

The many practical applications of the combination of ion beams and nanostructures warrants general investigations of the nanostructure - ion beam interaction, a topic that has therefore gained increased interest very recently \cite{borschel_ion-solid_2012,greaves_enhanced_2013,nietiadi_sputtering_2014,johannes_ion_2015,urbassek_sputter_2015}.  This dissertation adds to this growing field of nanostructure - ion beam interaction the discussion of three effects which are especially important in high fluence irradiation and dedicates a separate chapter to each. 
 
\begin{description}
  \item[\normalfont Chapter 3 - Sputtering of Nanowires] \hfill \\
  In the dissertation of Dr. C. Borschel \cite{borschel_ion-solid_2012} the program \emph{iradina} \cite{borschel_ion_2011} was developed and used to simulate the ion irradiation of nanostructures. It predicts an enhanced, diameter-dependent sputter yield in nanoparticles. This chapter discusses the simulation and compares its predictions with experimentally obtained diameter-dependent sputtering in nanowires. Some first results on the sputtering in the $Mn$ irradiation of $GaAs$ were obtained and published elsewhere \cite{johannes_enhanced_2014}. The results presented here are on $Ar$ irradiated $Si$ nanowires. They were obtained in close cooperation with Stefan Noack in his M.Sc. and also published in reference \cite{johannes_anomalous_2015}.
  \item[\normalfont Chapter 4 - High Doping Concentrations in Nanowires] \hfill \\
  The concentration of dopants does not follow a linear increase with the fluence of ions implanted for high fluences. It has already been observed in the early days of investigations into ion implantation that sputtering of the target will dynamically change its composition during the ion irradiation in addition to the intended change by incorporation of the ions within the target material \cite{moller_tridyn_1984,moller_tridyn-binary_1988,miyagawa_computer_1991,sigmund_alloy_1993,eckstein_oscillations_2000}. This effect is enhanced in nanostructures, first, since the sputtering is enhanced when compared to bulk samples as shown in the preceding chapter, but also since there is simply less material in the structure. Hence, the effect of removing material by sputtering becomes significant at lower fluencies. The presented results are acquired by nano-XRF performed on 175\,keV Mn$^+$ ion irradiated ZnO nanowires \cite{johannes_enhanced_2014}. They are discussed in comparison to a pseudo-dynamic simulation performed using results from \emph{iradina}.
  \item[\normalfont Chapter 5 - Plastic Flow in Silicon Nanowires] \hfill \\
  In the high ion fluence irradiated $Si$ nanowires an unexpected tendency of the nanowires to become shorter was observed. This chapter presents a dedicated investigation into this plastic deformation of $Si$ under ion irradiation which has been previously seen only in high energy ($\ge\,MeV$) ion irradiations \cite{volkert_stress_1991,trinkaus_viscoelastic_1995,hedler_amorphous_2004,hedler_boundary_2005}.
\end{description}
