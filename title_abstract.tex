\titlehead{Friedrich Schiller Universität Jena\\
PAF}
\subject{Dissertation}
\title{High-Fluence Ion Beam Irradiation of Semiconductor Nanowires}
\author{Andreas Johannes}
\date{July 2015}
\maketitle


%\renewcommand{\abstractname}{Zusammenfassung}




\vspace*{-4cm}\section*{Theses of the Dissertation}

\pagenumbering{roman}
\setcounter{page}{1}


\begin{enumerate}
\item{Semiconductor nanowires are a suitable model system to investigate the interaction between energetic ions and nanostructures.}
\item{The Sigmund sputtering model is a good approximation for the sputtering in nanostructures. It explains the qualitative structure-size dependence of the sputter yield.}
\item{Sputtering from nanostructures is comparable in experiments and simulations with the program \emph{iradina}, a Monte-Carlo simulation tool based on the binary collision approximation.}
\item{The nanowire diameter and/or ion energy dependent sputter yield from nanowires is maximum where the nanowire diameter is equal to the projected range of the ion in the nanowire material.}
\item{The redeposition from the substrate onto upstanding nanowires is negligible compared to the sputtering from the nanowires. The fluence of atoms redeposited from the substrate onto the nanowire can be estimated to be around $0.1\cdot SY \cdot \Phi$, with $SY$ the sputter yield from the substrate surface.}
\item{Sputtering in nanostructures leads to a non-linear increase in the doping concentration with high irradiated ion fluences. This can be quantified by nano-XRF.}
\item{Static simulations are useful to predict doping concentrations up to an ion fluence where 20\% of the total volume effected by the ion beam is sputtered. This fluence may be significantly lower in nanostructures than in bulk, because of the enhanced sputtering in nanostructures.}
\item{$Si$ nanowires show plastic deformation when irradiated with $100$ and $300\,keV\,Ar^+$ ions at room-temperature.} 
\item{The deformation of $Si$ nanowires by room temperature ion irradiation is not caused by point defects and is not oriented along the ion beam direction, the nanowires always become shorter.}
\item{A FIB system equipped with a micro-manipulator is a fun tool to manipulate and manufacture useful nanowire samples.}

\end{enumerate}

